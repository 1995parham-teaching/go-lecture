\documentclass[]{description}

\عنوان{شرح دوره}
\جزئیات‌درس{
  نام={برنامه‌نویسی با \متن‌لاتین{Golang}},
  ترم={تابستان ۱۴۰۱},
}

\begin{document}

\عنوان‌ساز

\قسمت{معرفی}

زبان برنامه‌نویسی \متن‌لاتین{Golang} شباهت بسیار زیادی به زبان برنامه‌نویسی \متن‌لاتین{C} داشته و هدف آن کاهش پیچدگی در توسعه برنامه‌ها است.
این زبان به طور گسترده برای پیاده‌سازی سرور‌های وب، برنامه‌های کاربردی و ابزارهای مدیریت کانتینرها مورد استفاده قرار گرفته است.
از جمله ابزارهایی که با این زبان توسعه پیدا کرده‌اند می‌توان به
\تارنما{https://github.com/kubernetes/kubernetes}{\متن‌لاتین{Kubernetes}}،
\تارنما{https://github.com/moby/moby}{\متن‌لاتین{Docker}}،
\تارنما{https://github.com/nats-io/nats-server}{\متن‌لاتین{NATS}}
و \تارنما{https://github.com/prometheus/prometheus}{\متن‌لاتین{Prometheus}} اشاره کرد.
در چند سال اخیر این زبان در شرکت‌های ایرانی هم جای خود را باز کرده است و از آن برای توسعه سرویس‌های بکند استفاده می‌شود.

\قسمت{سرفصل‌ها}

\شروع{فقرات}
\فقره تاریخچه
\فقره متغیرها و ثابت‌ها
\فقره محاسبات
\فقره شرط‌ها
\فقره حلقه‌ها
\فقره توابع
\فقره رشته‌ها
\فقره آرایه‌ها و \متن‌لاتین{slice}ها
\فقره \متن‌لاتین{map}ها
\فقره \متن‌لاتین{struct}ها
\فقره \متن‌لاتین{interface}ها
\فقره اشاره‌گرها
\فقره خطاها
\فقره همروی و کانال‌ها
\فقره \متن‌لاتین{select}
\فقره \متن‌لاتین{go module} و استفاده از بسته‌ها
\فقره مروری بر ویژگی‌های منحصر به فرد
\فقره آشنایی با پروتکل \متن‌لاتین{HTTP}
\فقره پیاده‌سازی سرور \متن‌لاتین{HTTP}
\فقره مدیریت تنظیمات
\فقره \متن‌لاتین{Metric}، \متن‌لاتین{Log} و \متن‌لاتین{Tracing}
\فقره ارتباط با پایگاه داده \متن‌لاتین{MongoDB}
\فقره آشنایی با \متن‌لاتین{Docker}
\پایان{فقرات}

در ابتدا دوره، یک آشنایی با زبان \متن‌لاتین{Go} ایجاد شده و دانشجویان برنامه‌های ساده‌ای را با آن پیاده‌سازی می‌کنند. از آنجایی که پیاده‌سازی سرورهای وب
یکی از موارد مهم استفاده از زبان برنامه‌نویسی \متن‌لاتین{Go} است، در ادامه مروری به ساختار پروتکل \متن‌لاتین{HTTP} کرده و پس از آن به پیاده‌سازی یک سرور وب ساده
در \متن‌لاتین{Go} می‌پردازیم. در این پیاده‌سازی سعی می‌شود آشنایی با ساختار برنامه‌های بزرگ در \متن‌لاتین{Go} ایجاد شده و جزئیایتی مانند \متن‌لاتین{Configuration} یا \متن‌لاتین{Metric}ها
که در سیستم‌های واقعی ارزش زیادی دارند مرور شود. در نهایت به این سرور وب یک پایگاه داده‌ای \متن‌لاتین{MongoDB} اضافه می‌شود که هدف از آن تنها آشنایی دانشجویان با رابط‌های پایگاه‌داده‌ای در زبان \متن‌لاتین{Go}
است و پرداخت زیادی به مباحث پایگاه‌داده‌ای صورت نمی‌پذیرد.
در نهایت برای بالا آوردن سیستم از \متن‌لاتین{Docker} و \متن‌لاتین{docker-compose} استفاده شده و سعی می‌شود مقدمه‌ای از اهمیت \متن‌لاتین{Docker} و چگونگی استفاده از آن ارائه شود.

\قسمت{مدت زمان}

برای مباحث اولیه با احتساب انجام پروژه‌های کوچک در زمان کلاس، به چیزی در حدود ۵۰ ساعت زمان احتیاج است.
در ادامه برای مباحث تخصصی سرور وب و پیاده‌سازی آن در حدود ۵۰ ساعت یا بیشتر زمان لازم است. این قسمت در کنار مفاهیم تئوری
زمان بیشتری برای پیاده‌سازی و تمرین در طول کلاس نیاز دارد.
در صورتی که دانشجویان قبلتر تجربه کار با هیچ زبان برنامه‌نویسی را نداشته باشند این زمان‌ها بیشتر خواهد شد و به تمرین بیشتری در کلاس احتیاج است.

\پایان‌ساز

\end{document}
